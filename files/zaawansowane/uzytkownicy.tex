\subsubsection{Dodawanie nowego użytkownika}
Aby dodać nowego użytkownika systemu uruchom w Dashu \textcolor{ubuntu_orange}{Konta użytkowników}. W prawym górnym rogu ekranu kliknij na \textcolor{ubuntu_orange}{Odblokuj} i podaj swoje hasło. Aby dodać nowego użytkownika:
\begin{enumerate}
\item Kliknij na przycisk \textcolor{ubuntu_orange}{+} znajdujący się pod lewym panelem z listą użytkowników.
\item Wypełnij formularz:
\begin{itemize}
\item \textcolor{ubuntu_orange}{Typ konta} --- konto standardowe umożliwia korzystanie z komputera. Konto administratora pozwala dokonywać zmian w systemie (instalacja pakietów, zmiana ustawień, wszędzie tam gdzie do tej pory podawałeś hasło aby coś odblokować).
\item \textcolor{ubuntu_orange}{Imię i nazwisko} --- aby system wiedział jak ma się zwracać do użytkownika
\item \textcolor{ubuntu_orange}{Nazwa użytkownika} --- systemowa nazwa, inaczej login do konta. Tylko małe litery, bez polskich znaków i znaków specjalnych.
\end{itemize}
\item Kliknij \textcolor{ubuntu_orange}{Dodaj}
\item Aby aktywować konto ustaw dla niego hasło. Kliknij na \textcolor{ubuntu_orange}{Brak} obok pola \textcolor{ubuntu_orange}{Hasło} w prawej części okna. Możesz tam też ustawić logowanie bez hasła.
\end{enumerate}

\subsubsection{Konto gościa}
Ubuntu umożliwia korzystanie z systemu w sesji gościa. Wyboru należy dokonać na ekranie logowania. Wszystkie zmiany wprowadzone podczas takiej sesji zostaną utracone po zakończeniu pracy. Gość nie może też dokonywać zmian w systemie, ma dostęp tylko do aktualnie zainstalowanego oprogramowania i nie może instalować nowego.

Aby wyłączyć konto gościa uruchom edytor tekstu jako administrator:
\begin{lstlisting}[language=bash]
sudo gedit
\end{lstlisting}

Wklej do niego:

\begin{lstlisting}[language=bash]
[SeatDefaults]
allow-guest=false
\end{lstlisting}

Zapisz plik jako /usr/share/lightdm/lightdm.conf.d/50-no-guest.conf\\
Aby ponownie włączyć konto gościa skasuj tamten plik:

\begin{lstlisting}[language=bash]
sudo rm /usr/share/lightdm/lightdm.conf.d/50-no-guest.conf
\end{lstlisting}