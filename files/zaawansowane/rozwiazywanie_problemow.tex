\label{rozwiązywanie problemów}
\subsubsection{Brak możliwości uruchomienia systemu Ubuntu lub Windows}
Zajrzyj do \ref{grub_przywracanie}: ,,Przywracanie GRUB-a''.

\subsubsection{GRUB startuje, ale nie uruchamia systemu}
W GRUB-ie wybierz \menu{{Opcje zaawansowane dla systemu Ubuntu}>{Recovery Mode}}. Odczekaj, aż system się załaduje, i wybierz opcję \textcolor{ubuntu_orange}{grub}. Następnie wybierz opcję \textcolor{ubuntu_orange}{resume} lub zrestartuj komputer.

\subsubsection{System plików został uszkodzony i wymaga naprawy}
W GRUB-ie wybierz \menu{{Opcje zaawansowane dla systemu Ubuntu}>{Recovery Mode}}. Odczekaj, aż system się załaduje, i wybierz opcję \textcolor{ubuntu_orange}{fsck}. Następnie wybierz opcję \textcolor{ubuntu_orange}{resume} lub zrestartuj komputer.

\subsubsection{Czarny ekran podczas uruchamiania systemu}
Przyczyny tego mogą być różne, jednak najczęściej jest to problem ze sterownikami karty graficznej. Po pierwsze karta graficzna mogła nie powiadomić monitora, że ma zacząć wyświetlać obraz. Wciśnij \keys{CTRL + Alt + F1} aby przejść na wirtualną konsolę a następnie \keys{CTRL + Alt + F7} aby powrócić do trybu graficznego.

Jeżeli to nie pomaga, uruchom ponownie komputer, wejdź w menu GRUB-a i wybierz \menu{{Opcje zaawansowane dla systemu Ubuntu}>{Recovery Mode}}. Odczekaj, aż system się załaduje, i wybierz opcję \textcolor{ubuntu_orange}{failsafex}. System zostanie uruchomiony w bezpiecznym trybie graficznym. Wybierz \textcolor{ubuntu_orange}{Run in low-graphics mode for just one session}. Przejdź do sekcji \ref{sterowniki} ,,Sterowniki do kart graficznych'', aby dowiedzieć się więcej o sterownikach.

Trzecią możliwością jest dopisanie opcji ,,nomodeset'' do parametrów uruchamiania Ubuntu. W~menu GRUB-a zaznacz Ubuntu i wciśnij \keys{e}. Przejdź do linijki zaczynającej się od \textcolor{ubuntu_orange}{linux}	(to będzie przedostatnia linijka), skasuj \$Vt\_handoff i na końcu dopisz \textcolor{ubuntu_orange}{nomodeset}. Wciźnij \keys{CTRL + x}, aby uruchomić komputer z nowymi parametrami.

\subsubsection{Problemy z instalacją pakietów}
Przerwany proces instalacji pakietów oprogramowania skutkuje niemożliwością dokonywania zmian w systemie. Menadżer pakietów w takiej sytuacji wykrywa, że pakiety zostały uszkodzone i nie pozwoli nic zrobić, póki nie zostaną naprawione. Najprostszym remedium jest uruchomienie konsoli \keys{CTRL + Alt + t} i wpisanie następujących poleceń:

\begin{lstlisting}[language=bash]
sudo dpkg --configure -a
sudo apt-get clean
sudo apt-get autoclean
sudo apt-get install -f
\end{lstlisting}

\subsubsection{Przywracanie paska launchera}
Aby przywrócić domyślne ikony na pasku Launchera, wykonaj:

\begin{lstlisting}[language=bash]
unity --reset-icons
\end{lstlisting}

Zmiany będą widoczne po ponownym zalogowaniu się do systemu.

\subsubsection{Przywracanie wyglądu systemu do ustawień domyślnych}
\label{unity_reset} \noindent Aby przywrócić domyślny wygląd interfejsu użytkownika, wykonaj w terminalu:

\noindent Samo Unity:
\begin{lstlisting}[language=bash]
unity --reset
\end{lstlisting}

\noindent Przywracanie domyślnych ustawień całego środowiska graficznego:

\begin{lstlisting}[language=bash]
sudo apt-get install dconf-tools
dconf reset -f /org/compiz/
\end{lstlisting}

\noindent Zmiany będą widoczne po ponownym zalogowaniu się do systemu.
\clearpage
