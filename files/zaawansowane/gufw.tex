O tym, czy firewall w systemie Ubuntu jest potrzebny, czy nie, można długo dysktuować. Jeżeli dla spokoju ducha potrzebujesz działającej ,,ściany ogniowej'', to Ubuntu oferuje proste narzędzie do zarzadzania wbudowanym w system programem iptables.

GUFW (\textcolor{ubuntu_orange}{GUI for Uncomplicated FireWall} znajduje się w repozytoriach Ubuntu. Zainstaluj go z Centrum Oprogramowania (,,Konfiguracja zapory Firewall'') lub za pomocą terminala:
\begin{lstlisting}[language=bash]
sudo apt-get install gufw
\end{lstlisting}

\begin{wrapfigure}{R}{0.5\textwidth}
	\vspace{-10pt}
	\includegraphics[width=\linewidth]{images/programy_gufw.png}
\end{wrapfigure}

Uruchom program GUFW. Konfiguracja zapory sieciowej wymaga uprawnień administratora, zostaniesz więc poproszony o uwierzytelnienie. Kiedy to zrobisz, otwarte zostanie główne okno programu.

Domyślnie zapora jest wyłączona. Kliknij przycisk przy pozycji \textcolor{ubuntu_orange}{Stan} i ustaw go na włączony (tak jak na rysunku). Od tego momentu firewall działa i będzie automatycznie się uruchamiać wraz ze startem systemu. Warto jeszcze przestawić \textcolor{ubuntu_orange}{Przychodzące} z pozycji ,,Odmów'' na ,,Odrzuć''\footnote{Kiedy ktoś będzie próbował się połączyć z zewnątrz, to zamiast otrzymać sygnał ,,zajęty'', dostanie ,,głuchy telefon''.}. Teraz firewall jest skonfigurowany według ogólnie przyjętego na świecie schematu dla komputerów osobistych: odrzuć wszystkie połączenia przychodzace, zezwól lokalnym programom na łączenie się z internetem.

\subsubsection{Dodawanie reguł}
Może się zdażyć, iż będziesz potrzebował wprowadzić jakiś wyjątek od wcześniej ustalonej polityki działania firewalla. Zrobisz to w zakładce \textcolor{ubuntu_orange}{Reguły}. Aby dodać regułę, kliknij przycisk +, znajdujący się pod tabelą. Są trzy sposoby na stworzenie reguły:
\begin{itemize}
\item \textcolor{ubuntu_orange}{Predefiniowana} --- pozwala stworzyć regułę w oparciu o szablon. Jeżeli chcesz zablokować lub odblokować konkretny program, wybierz go z listy \textcolor{ubuntu_orange}{Programy} i ustaw odpowiednią metodę.
\item \textcolor{ubuntu_orange}{Prosta} --- pozwala w prosty sposób zarzadzać poszczególnymi protokołami. Na przykład, aby zablokować połączenia przeglądarek internetowych (i wielu innych programów), wpisz \textit{http} w pole \textcolor{ubuntu_orange}{Port} i wybierz ,,Odrzuć'' z listy \textcolor{ubuntu_orange}{Metoda}.
\item \textcolor{ubuntu_orange}{Prosta} --- pozwala ograniczyć dostep do specyficznych adresów IP, protokołów, czy konkretnych typów połączeń z siecią (np. ethernet, WiFi).
\end{itemize}
