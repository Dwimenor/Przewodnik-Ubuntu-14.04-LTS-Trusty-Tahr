Domyślną przeglądarką internetową w systemie Ubuntu jest Firefox. Przeglądarka ta pracuje i zachowuje się tak samo jak swoje wersje na systemy Mac czy Windows. Dostępne sa te same rozszerzenia, motywy i ustawienia.

\subsubsection{Przenoszenie ustawień}
Firefox przechowuje swoje ustawienia w katalogu .mozilla znajdującym się w twoim katalogu domowym\footnote{ukryte pliki i katalogi (zaczynające się od kropki) wyświetlisz wciskając \keys{CTRL + h}}. Zrobienie kopii zapasowej tego katalogu umożliwi przeniesienie ustawień pomiędzy różnymi systemami operacyjnymi. W innych systemach operacyjnych ten katalog znajduje się w :
\begin{itemize}
\item Windows 95, 98, ME\\
\directory{C / Windows / {Application Data} / Mozilla}\\
\directory{C / Windows/ Profiles / {Nazwa użytkownika} / {Application Data} / Mozilla}
\item Windows 2000, XP\footnote{Aby wyświetlić ukryte pliki \menu{Narzędzia>Opcje Folderów>Widok>{Pokazuj ukryte pliki i foldery}}}\\
\directory{C / {Documents and Settings}  / {Nazwa użytkownika} / {Application Data} / Mozilla}
\item Windows Vista, 7\footnote{Patrz wyżej}\\
\directory{C / Users / {Nazwa użytkownika} / AppData / Roaming / Mozilla}
\item Mac\\
\directory{\textasciitilde / Library / {Application Support} / Firefox}\\
\directory{\textasciitilde / Library / Mozilla / Firefox}
\end{itemize}

\subsubsection{Flash}
Wtyczka flash została zainstalowana z pakietem \textcolor{ubuntu_orange}{ubuntu-restricted-extras}. Jeżeli chcesz zainstalować samą wtyczkę flash, bez innych dodatków, potrzebujesz pakietu \textcolor{ubuntu_orange}{flashplugin-installer}. Niestety, Adobe zaprzestało rozwoju wtyczki Flash dla Linuksa i wersja instalowana w ten sposób jest stara. Do większości zadań wystarczy, ale wiele internetowych gier wymaga nowszej wersji wtyczki. Rozwiązania są dwa. Ablo instalacja przeglądarki Google Chrome (ta przeglądarka dystrybuowana jest z najnowszą wersją Flasha wbudowaną w nią) albo instalacja Windowsowego pluginu za pośrednictwem aplikacji Pipelight. Zostało to omówione w następnej sekcji.

\subsubsection{Instalacja Pipelight}
Aby móc odtwarzać materiały przy pomocy wtyczek Silverlight, Widevine, ViewRight Caiway czy najnowszy Flash, potrzebujesz aplikacji PipeLight. Aby ją zainstalować wykonaj następujące polecenia w konsoli\footnote{konsolę otworzysz kombinacją klawiszy \keys{CTRL + ALT + t}}
\begin{enumerate}
\item Dodanie repozytorium z programem PipeLight:
\begin{lstlisting}[language=bash]
sudo apt-add-repository ppa:pipelight/stable
\end{lstlisting}
\item Aktualizacja informacji o pakietach:
\begin{lstlisting}[language=bash]
sudo apt-get update
\end{lstlisting}
\item Instalacja programu:
\begin{lstlisting}[language=bash]
sudo apt-get install pipelight-multi
\end{lstlisting}
\end{enumerate}

\noindent Nie instaluj wszystkich dostępnych pluginów, gdyż znacznie spowolni to przeglądarkę. Używaj tylko tych co potrzebujesz.

Wiele stron internetowych sprawdza także zgodność przeglądarki z ich wymaganiami. Domyślnie odrzucają one nawet zgodne przeglądarki z odpowiednimi pluginami jeżeli wykryją, że pracują na systemie Linux. Aby oszukać takie strony potrzebujesz rozszerzenia zmieniającego wartość \textcolor{ubuntu_orange}{UserAgent} wysyłaną do strony internetowej. Dla Firefoka polecany jest dodatek \href{https://addons.mozilla.org/pl-PL/firefox/addon/user-agent-overrider/}{User Agent Overrider}. Po zainstalowaniu tego dodatku udaj się do \menu{{Narzędzia}>{Dodatki}>{Rozszerzenia}>{User Agent Overrider}>{Preferencje}} i wklej na końcu otwartego okna poniższe dwie linijki:
\begin{itemize}
\item Firefox 15/Windows: Mozilla/5.0 (Windows NT 6.1; WOW64; rv:15.0) Gecko/20120427 Firefox/15.0a1
\item Safari/OSX: Mozilla/5.0 (Macintosh; Intel Mac OS X 10\_7\_3) AppleWebKit/534.55.3 (KHTML, like Gecko) Version/5.1.3 Safari/534.53.10
\end{itemize}

Kiedy natrafisz na problematyczną stronę to zmień UserAgent i przeładuj ją. Przycisk pozwalający zmienić ustawienia znajduje się na prawo od paska adresu w oknie przeglądarki.

\subsubsection{Instalacja wtyczki Silverlight}
\noindent Jak masz już zainstalowany program PipeLight to wykonaj:
\begin{enumerate}
\item Zainstaluj wtyczkę Silverlight:
\begin{lstlisting}[language=bash]
sudo pipelight-plugin --enable silverlight
\end{lstlisting}
\item Zostaniesz poproszony o zaakceptowanie warunków licencji. Wciśnij \keys{y}
\item Zresetuj przeglądarkę Firefox.
\item Po ponownym uruchomieniu przeglądarki pobrane i zainstalowane zostaną potrzebne pliki. Może to potrwać kilkanaście sekund. W tym czasie przeglądarka bedzie niedostępna.
\item Wejdź na stronę \href{http://bubblemark.com/silverlight2.html}{testową}. Jeżeli wszystko zostało zainstalowane poprawnie to wtyczka Silverlight wyświetli zielone, skaczace kulki.
\end{enumerate}

\subsubsection{Instalacja wtyczki Flash}
\noindent Jak masz już zainstalowany program PipeLight to wykonaj:
\begin{enumerate}
\item Usuń starą wtyczkę Flash:
\begin{lstlisting}[language=bash]
sudo apt-get remove --purge flashplugin-installer
\end{lstlisting}
\item Zainstaluj nową wersję Flasha za pomocą PipeLight:
\begin{lstlisting}[language=bash]
sudo pipelight-plugin --enable flash
\end{lstlisting}
\item Zostaniesz poproszony o zaakceptowanie warunków licencji. Wciśnij \keys{y}
\item Zresetuj przeglądarkę Firefox.
\item Po ponownym uruchomieniu przeglądarki pobrane i zainstalowane zostaną potrzebne pliki. Może to potrwać kilkanaście sekund. W tym czasie przeglądarka bedzie niedostępna.
\item Wejdź na stronę \href{https://www.adobe.com/software/flash/about/}{testową}. Jeżeli wszystko zostało zainstalowane poprawnie to zobaczysz animowane logo Adobe Flash.
\end{enumerate}

\subsubsection{Instalacja wtyczki Shockwave Player}
\noindent Jak masz już zainstalowany program PipeLight to wykonaj:
\begin{enumerate}
\item Zainstaluj wtyczkę:
\begin{lstlisting}[language=bash]
sudo pipelight-plugin --unlock shockwave
sudo pipelight-plugin --enable shockwave
\end{lstlisting}
\item Zostaniesz poproszony o zaakceptowanie warunków licencji. Wciśnij \keys{y}
\item Zresetuj przeglądarkę Firefox.
\item Po ponownym uruchomieniu przeglądarki pobrane i zainstalowane zostaną potrzebne pliki. Może to potrwać kilkanaście sekund. W tym czasie przeglądarka bedzie niedostępna.
\item Jeżeli na stronach wykorzystujących Shockwave Player masz puste prostokąty, kliknij na jednym z nich prawym przyciskiem myszy i wybierz \textcolor{ubuntu_orange}{OpenGL} jak silnik renderujący.
\end{enumerate}

\subsubsection{Instalacja wtyczki Unity Webplayer}
\noindent Jak masz już zainstalowany program PipeLight to wykonaj:
\begin{enumerate}
\item Zainstaluj wtyczkę:
\begin{lstlisting}[language=bash]
sudo pipelight-plugin --enable unity3d
\end{lstlisting}
\item Zostaniesz poproszony o zaakceptowanie warunków licencji. Wciśnij \keys{y}
\item Zresetuj przeglądarkę Firefox.
\item Po ponownym uruchomieniu przeglądarki pobrane i zainstalowane zostaną potrzebne pliki. Może to potrwać nawet lika minut. W tym czasie przeglądarka bedzie niedostępna.
\item Wejdź na stronę \href{https://unity3d.com/showcase/live-demos}{testową} aby przetestować działanie wtyczki.
\end{enumerate}

\subsubsection{Instalacja wtyczki Widevine do obsługi materiałów DRM}
\noindent Jak masz już zainstalowany program PipeLight to wykonaj:
\begin{enumerate}
\item Upewnij się, że masz zainstalowanego Flasha. Wtyczka Widavine działa zarówno z wtyczką linuksową jak wtyczką instalowaną przez PipeLight.
\item Zainstaluj wtyczkę:
\begin{lstlisting}[language=bash]
sudo pipelight-plugin --enable widevine
\end{lstlisting}
\item Zostaniesz poproszony o zaakceptowanie warunków licencji. Wciśnij \keys{y}
\item Zresetuj przeglądarkę Firefox.
\item Po ponownym uruchomieniu przeglądarki pobrane i zainstalowane zostaną potrzebne pliki. Może to potrwać nawet lika minut. W tym czasie przeglądarka bedzie niedostępna.
\item Wejdź na stronę \href{http://www.widevine.com/demo/index.html}{testową} aby przetestować działanie wtyczki.
\end{enumerate}

\subsubsection{Instalacja innych pluginów}
\noindent Pipelight umożliwia instalację kilku dodatkowych pluginów. Ich aktywowanie przebiega zawsze w identyczny sposób:
\begin{enumerate}
\item Zainstaluj wtyczkę:
\begin{lstlisting}[language=bash]
sudo pipelight-plugin --enable nazwa_wtyczki
\end{lstlisting}
\item Zostaniesz poproszony o zaakceptowanie warunków licencji. Wciśnij \keys{y}
\item Zresetuj przeglądarkę Firefox.
\item Po ponownym uruchomieniu przeglądarki pobrane i zainstalowane zostaną potrzebne pliki. Może to potrwać kilkanaście sekund. W tym czasie przeglądarka bedzie niedostępna.
\end{enumerate}
Lista dostępnych pluginów:
\begin{itemize}
\item \textcolor{ubuntu_orange}{adobereader} - Zastępuje wbudowaną w Firefoksa przeglądarkę PDF pluginem Adobe Acrobat Reader.
\item \textcolor{ubuntu_orange}{foxitpdf} - Zastępuje wbudowaną w Firefoksa przeglądarkę PDF pluginem Foxit Reader.
\item \textcolor{ubuntu_orange}{grandstream} - Plugin do podłączania kamer IP.
\item \textcolor{ubuntu_orange}{hikvision} - Plugin do podłączania kamer IP.
\item \textcolor{ubuntu_orange}{roblox} - Plugin do gier napisanych na silniku roblox.
\end{itemize}

\subsubsection{Rozwiązywanie problemów}
Jeżeli z jakiegokolwiek powodu przeglądarka zaczyna źle działac to pierwszym etapem jest przywrócenie ustawień domyślnych. W ten sposób upewnisz się, że nieprawidłowe działanie przeglądarki spowodowane jest złymi ustawieniami użytkownika (np. wadliwy dodatek) a nie kwestią systemową.
\begin{enumerate}
\item Wyłącz przeglądarkę Firefox
\item Przenieś katalog .mozilla w bezpieczne miejsce (wciśnij \keys{CTRL + h} aby wyświetlić ukryte pliki) lub wykonaj:
\begin{lstlisting}[language=bash]
mv ~/.mozilla ~/.mozilla_bak
\end{lstlisting}
\begin{itemize}
\item \textcolor{ubuntu_orange}{mv} - polecenie przenoszenia plików move.
\item \textcolor{ubuntu_orange}{\textasciitilde /.mozilla} który plik ma zostać przeniesiony.
	\begin{description}
	\item[\textasciitilde /] - tylda i slash oznaczają katalog domowy użytkownika
	\item[.mozilla] katalog do skopiowania.
	\end{description}
\item \textasciitilde / .mozilla.bak - miejsce docelowe.
\end{itemize}
\item Włącz ponownie przeglądarkę i sprawdź czy problem dalej występuje.
\end{enumerate}