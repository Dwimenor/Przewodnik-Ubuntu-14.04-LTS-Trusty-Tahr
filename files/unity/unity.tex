\input{unity}
W momencie pierwszego uruchomienia, zaraz po instalacji systemu, można zauważyć, że wygląd Ubuntu wcale aż tak bardzo nie różni się od wyglądu systemów Microsoft Windows, czy Mac OS X. Wszystkie systemy łączy ta sama idea graficznego interfejsu (\textcolor{ubuntu_orange}{GUI: Graphical User Interface}), w którym użytkownik systemu za pomocą myszy kontroluje większość zadań, począwszy od zwykłego poruszania się po systemie, przez otwieranie dostępnych aplikacji, aż po tworzenie, usuwanie oraz przenoszenie plików i katalogów.

W Ubuntu domyślnym środowiskiem graficznym jest Unity.

W momencie pierwszego uruchomienia, zaraz po instalacji systemu, można zauważyć, że wygląd Ubuntu wcale aż tak bardzo nie różni się od wyglądu systemów Microsoft Windows, czy Mac OS X. Wszystkie systemy łączy ta sama idea graficznego interfejsu (\textcolor{ubuntu_orange}{GUI: Graphical User Interface}), w którym użytkownik systemu za pomocą myszy kontroluje większość zadań, począwszy od zwykłego poruszania się po systemie, przez otwieranie dostępnych aplikacji, aż po tworzenie, usuwanie oraz przenoszenie plików i katalogów.

W Ubuntu domyślnym środowiskiem graficznym jest Unity.

W momencie pierwszego uruchomienia, zaraz po instalacji systemu, można zauważyć, że wygląd Ubuntu wcale aż tak bardzo nie różni się od wyglądu systemów Microsoft Windows, czy Mac OS X. Wszystkie systemy łączy ta sama idea graficznego interfejsu (\textcolor{ubuntu_orange}{GUI: Graphical User Interface}), w którym użytkownik systemu za pomocą myszy kontroluje większość zadań, począwszy od zwykłego poruszania się po systemie, przez otwieranie dostępnych aplikacji, aż po tworzenie, usuwanie oraz przenoszenie plików i katalogów.

W Ubuntu domyślnym środowiskiem graficznym jest Unity.

W momencie pierwszego uruchomienia, zaraz po instalacji systemu, można zauważyć, że wygląd Ubuntu wcale aż tak bardzo nie różni się od wyglądu systemów Microsoft Windows, czy Mac OS X. Wszystkie systemy łączy ta sama idea graficznego interfejsu (\textcolor{ubuntu_orange}{GUI: Graphical User Interface}), w którym użytkownik systemu za pomocą myszy kontroluje większość zadań, począwszy od zwykłego poruszania się po systemie, przez otwieranie dostępnych aplikacji, aż po tworzenie, usuwanie oraz przenoszenie plików i katalogów.

W Ubuntu domyślnym środowiskiem graficznym jest Unity.
