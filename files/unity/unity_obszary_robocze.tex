Obszary robocze, powszechnie zwane wirtualnymi pulpitami, pozwalają w łatwy sposób zarządzać i grupować programy. Dzięki temu, przy dużej liczbie otwartych okien, w łatwy sposób możemy zredukować zatłoczenie i usprawnić nawigację po pulpicie. Dla przykładu, w jednym obszarze roboczym możemy przeglądać zasoby Internetu, a w drugim lub trzecim możemy mieć uruchomiony program do przeglądania multimediów oraz tworzyć dokument w programie biurowym.

Domyślnie Ubuntu daje dostęp do czterech obszarów roboczych, ale zaraz po instalacji ta opcja nie jest włączona. Aby móc korzystać z dodatkowych obszarów roboczych, kliknij ikonę systemową \includegraphics{images/ikony_zasilanie.png} na panelu menu (w prawym górnym rogu) i wybierz \menu{{Ustawienia Systemu}>{Osobiste}>{Wygląd}>{Zachowanie}>{Uaktywnij obszary robocze}}.

\begin{wrapfigure}{l}{0.05\textwidth}
	\vspace{-10pt}
	\includegraphics[width=\linewidth]{images/ikony_obszary_robocze.png}
\end{wrapfigure}

Po zaznaczeniu dodatkowych obszarów roboczych na pasku Launchera pojawi się dodatkowa ikona, służąca do przełączania pomiędzy obszarami roboczymi: \textcolor{ubuntu_orange}{Przełącznik obszarów roboczych}.

Domyślnie w Ubuntu masz do dyspozycji cztery wirtualne pulpity, ułożone w siatkę 2x2. Po naciśnięciu ikony obszarów roboczych na ekranie pojawia się siatka czterech pulpitów. Pomiędzy obszarami roboczymi możesz się też przemieszczać za pomocą następujących skrótów klawiszowych:
\begin{itemize}
\item \keys{CTRL + ALT + \arrowkeyright} przenosi do obszaru roboczego po prawej;
\item \keys{CTRL + ALT + \arrowkeyleft} przenosi do obszaru roboczego po lewej;
\item \keys{CTRL + ALT + \arrowkeyup} przenosi do obszaru roboczego na górze;
\item \keys{CTRL + ALT + \arrowkeydown} przenosi do obszaru roboczego na dole.
\end{itemize}
