\begin{wrapfigure}{l}{0.1\textwidth}
	\vspace{-10pt}
	\includegraphics[width=\linewidth]{images/ikony_empathy.png}
\end{wrapfigure}

\textcolor{ubuntu_orange}{Empathy} jest wbudowanym w Ubuntu komunikatorem internetowym, obsługującym różne protokoły sieciowe. Do nawiązywania rozmów Empathy domyślnie korzysta z systemowej usługi \textcolor{ubuntu_orange}{Konta sieciowe}.

Poza usługami, z którymi możesz się połączyć przez Konta sieciowe, Empathy obsługuje protokoły Jabbera, AIM, Gadu-Gadu, GroupWize, ICQ, IRC, MSN (Windows Live) i inne.

\subsubsection{Sterowanie Empathy}
Empathy można sterować za pomocą ikony \includegraphics{images/unity_wskaznik_wiadomosci.png}, umieszczonej na pasku menu. Wskaźnik zmienia kolor na niebieski, jeżeli są jakieś nieprzeczytane wiadomości. Kliknięcie tego wskaźnika otwiera menu pozwalające zmienić status (np. dostępny, niewidoczny, zajęty, rozłączony), a także wyświetla ewentualne rozpoczęte, ale nieprzeczytane rozmowy. Pozycja \textcolor{ubuntu_orange}{Empathy} uruchamia główne okno programu, zawierające listę kontaktów.

\subsubsection{Przykładowa konfiguracja konta Gadu-Gadu}
\begin{center}
	\includegraphics[width=\linewidth]{images/programy_empathy1.png}
\end{center}
\begin{enumerate}
\item W Dashu wyszukaj i uruchom \textcolor{ubuntu_orange}{empathy-accounts}.
\item Zostaniesz poproszony o wpisanie swoich danych do komunikacji wewnątrz sieci lokalnej. Możesz pominać ten krok.
\item Pod lewym panelem znajduje się przycisk \textcolor{ubuntu_orange}{+}. Kliknij go.
\item Z rozwijanej listy wybierz \textcolor{ubuntu_orange}{Gadu-Gadu}. W polu \textcolor{ubuntu_orange}{Konto} wpisz swój numer w sieci GG, a następnie kliknij \textcolor{ubuntu_orange}{Zastosuj}.
\item Konto zostało dodane. Wybierz \menu{{Modyfikuj parametry połączenia}>{Zaawansowane}} i wpisz swoje hasło. Kliknij \textcolor{ubuntu_orange}{Zastosuj}, aby zapisać hasło.
\item Połączenie z Gadu-Gadu jest gotowe.
\end{enumerate}
\clearpage