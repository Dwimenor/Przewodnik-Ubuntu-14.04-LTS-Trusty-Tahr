\begin{wrapfigure}{l}{0.1\textwidth}
	\vspace{-10pt}
	\includegraphics[width=\linewidth]{images/ikony_libreoffice.png}
\end{wrapfigure}

LibreOffice to pakiet biurowy, dostępny po zainstalowaniu systemu Ubuntu. W skład pakietu wchodzi następujące oprogramowanie:
\begin{itemize}
\item \textcolor{ubuntu_orange}{Writer} --- procesor tekstu;
\item \textcolor{ubuntu_orange}{Calc} --- arkusz kalkulacyjny;
\item \textcolor{ubuntu_orange}{Impress} --- program do przygotowywania prezentacji;
\item \textcolor{ubuntu_orange}{Draw} --- program do grafiki wektorowej, przygotowywania schematów, itp.;
\item \textcolor{ubuntu_orange}{Math} --- kreator równań matematycznych;
\item \textcolor{ubuntu_orange}{Base} --- graficzny system zarzaqdzania bazami danych.
\end{itemize}
