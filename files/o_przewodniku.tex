Niniejszy przewodnik powstał przy wspólnym wysiłku użytkowników \href{http://ubuntu.pl/forum/}{forum.ubuntu.pl}. Mamy nadzieję, że będzie on pomocny w stawianiu pierwszych kroków z systemem Ubuntu Linux. W razie jakichkolwiek kłopotów zawsze możesz poprosić o pomoc na naszym forum.

\subsection{Autorzy}
Główni autorzy przewodnika:
\begin{itemize}
\item Piotr "Dwimenor Vali" Sochocki, dwimeron@gmail.com --- tekst, skład;
\item makson --- tekst, korekta;
\item bear 7 --- tekst, korekta;
\item Ionash --- korekta;
\item K2Cl --- korekta;
\item perzan --- pomoc przy składzie.
\end{itemize}

Dodatkowe podziękowania za korektę i wyłapywanie literówek:\\
babel-89, enedil, kszyhus, qnebra, rob006, Tomfoc

\subsection{Linki}
\begin{itemize}
\item \href{http://ubuntu.pl/dokumenty/Przewodnik_Ubuntu_14.04_LTS_Trusty_Tahr.pdf}Najnowsza wersja \textit{Przewodnika} (ubuntu.pl)
\item \href{https://github.com/Dwimenor/Przewodnik-Ubuntu-14.04-LTS-Trusty-Tahr}{Źródła} (github.com)
\end{itemize}

\subsection{Wykorzystane materiały}
W \textit{Przewodniku} wykorzystano dodatkowe, zewnętrzne materiały:
\begin{itemize}
\item Strona \pageref{shotwell}: Galeria zdjęć Beshefa (\href{https://www.flickr.com/photos/sharif/sets/72157600223985436/}{link});
\item Strona \pageref{totem}: kadr z filmu \href{http://www.bigbuckbunny.org}{Big Buck Bunny};
\item Strona \pageref{polecenia}: ,,Informator poleceń systemu Unix/Linux'', oryginalnie napisany dla serwisu \href{http://fosswire.com/}{Fosswire.com}, udostępniony na licencji CC-by-sa 3.0. Polskie tłumaczenie dostępne jest \href{http://czytelnia.ubuntu.pl/index.php/2012/02/24/wydrukuj-i-powies-kolo-monitora-informator-polecen-systemu-unixlinux/}{pod tym linkiem}.
\item Ikony programów: domyślny styl ikon Ubuntu (ubuntu-mono-dark);
\item Szata graficzna: \href{http://design.ubuntu.com/}{design.ubuntu.com}.
\end{itemize}

\subsection{Licencja}
\begin{wrapfigure}{l}{0.1\textwidth}
	\vspace{-10pt}
	\includegraphics[width=\linewidth]{images/CC.png}
\end{wrapfigure}

,,Przewodnik po Ubuntu 14.04 LTS Trusty Tahr'' jest objęty licencją [\href{http://creativecommons.org/licenses/by-nc-sa/3.0/pl/}{Creative Commons Uznanie autorstwa -- Użycie niekomercyjne -- Na tych samych warunkach 3.0 Polska}.

Skrócone zapisy licencji:\\
Zezwala się na kopiowanie i rozpowszechnianie \textit{Przewodnika} w dowolnym medium i formacie a także na jego adaptację (remiksowanie, zmienianie i tworzenie utworów zależnych) na następujących warunkach:
\begin{itemize}
\item Uznanie autorstwa --- Należy odpowiednio oznaczyć autorstwo utworu, podać odnośnik do licencji i wskazać zmiany, jeśli takie zostały dokonane. Można tego dokonać w dowolny rozsądny sposób, o ile nie sugeruje się udzielenia przez licencjodawcę aprobaty dla siebie lub swojego sposobu wykorzystania licencjonowanego utworu.
\item Użycie niekomercyjne --- Nie należy wykorzystywać utworu do celów komercyjnych.
\item Na tych samych warunkach --- Remiksując utwór, przetwarzając go lub tworząc na jego podstawie, należy swoje dzieło rozpowszechniać na tej samej licencji, co oryginał. 
\end{itemize}
\clearpage