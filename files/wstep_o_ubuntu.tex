Ubuntu jest kompletnym systemem operacyjnym utrzymywanym i rozwijanym przez firmę Canonical. Pierwsza jego wersja ukazała się w 2004 roku, a w ciągu 10 lat system ten zdobył rzesze fanów. Ubuntu wraz ze swoimi odmianami jest najpopularniejszą na świecie dystrybucją Linuksa. Samo słowo Ubuntu w języku afrykańskiego plemienia Zulusów oznacza ,,człowieczeństwo wobec innych'', w kontekście systemu operacyjnego tłumaczone jest jednak jako ,,Linux dla ludzi''.

Ideą systemu Ubuntu jest dostarczenie użytkownikowi kompletnego systemu operacyjnego, zawierającego wszystkie elementy niezbędne do codziennej pracy, a jednocześnie umożliwiającego posiadaczowi komputera swobodne korzystanie z systemu i modyfikowanie poszczególnych jego elementów. Wybierając Ubuntu nie musisz się zastanawiać nad tym, czy twój procesor nie ma przypadkiem zbyt dużej liczby rdzeni, co w przypadku korzystania z systemu komercyjnego mogłoby wymagać zakupu innej licencji. Nie musisz się również przejmować tym, że w firmie masz dziesięć komputerów, a twoja licencja na pakiet biurowy pozwala na instalację jedynie na sześciu stanowiskach. Jeśli chodzi o to, jak i do czego wykorzystasz system i oprogramowanie, wszystko zależy wyłącznie od ciebie.

Ubuntu pozwala także na daleko idące modyfikacje systemu, gdyż jego kod źródłowy jest otwarty. Choć powyższe zdanie może brzmieć groźnie, nie ma powodów do obaw --- Ubuntu nie jest przeznaczone tylko dla doświadczonych komputerowych ,,,magików'''. Każdy może dowolnie dostosować swój system do własnych potrzeb i upodobań, czy to metodą ,,zrób to sam'', czy też poprzez odwołanie się do zasobów oferowanych przez społeczność.

Skoro poruszyliśmy już ten temat --- społeczność skupiona wokół Ubuntu jest najważniejszą siłą napędzającą rozwój tej dystrybucji. Dodatki zmieniające wygląd systemu, nowe ikony i grafiki, dźwięki systemowe, tłumaczenia, całe zestawy oprogramowania --- wszystkie te elementy (oraz wiele innych rzeczy) czekają, aż zdecydujesz się z nich skorzystać.
