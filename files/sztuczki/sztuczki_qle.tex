Kliknięcie prawym przyciskiem myszy ikony programu w Launcherze daje dostęp do Szybkiej Listy (QuickList) z dodatkowymi poleceniami, oferowanymi przez ten program. Przykłądowo, lista pakietu biurowego LibreOffice zawiera skróty, umożliwiające bezpośrednie otwarcie procesora tekstu Writer, akrusza kalkulacyjnego Calc, prezentacji Impress i innych.

Dzięki programowi QuickList Editor możliwe jest zarzadzanie zawartością Szybkich List --- dodawanie nowych poleceń, usuwanie starych i modyfikowanie istniejących. Aby zainstalować QLE, uruchom \menu{{Oprogramowanie i Aktualizacje}>{Inne oprogramowanie}>{Dodaj\ldots}}. W pole \textcolor{ubuntu_orange}{Wiersz APT} wpisz:
\begin{lstlisting}
http://ppa.launchpad.net/vlijm/qle/ubuntu
\end{lstlisting}
Kliknij przycisk \textcolor{ubuntu_orange}{Dodaj zasób} i potwierdź swoim hasłem. Kiedy spróbujesz wyjść z tego programu, zostaniesz poinformowany o potrzebie uaktualnienia listy pakietów. Zrób to i poczekaj chwilę. Kiedy operacja zostanie zakończona, otwórz Terminal i zainstaluj program QLE:
\begin{lstlisting}[language=bash]
sudo apt-get install qle
\end{lstlisting}
\begin{itemize}
\item \textcolor{ubuntu_orange}{sudo} --- wykonuje dalsze polecenia z uprawieniami administratora systemu;
\item \textcolor{ubuntu_orange}{apt-get} --- program do zarządzania zainstalowanym oprogramowaniem;
\item \textcolor{ubuntu_orange}{install} --- informujesz apt, że chcesz zainstalować paczkę z oprogramowaniem;
\item \textcolor{ubuntu_orange}{qle} --- nazwa paczki do zainstalowania.
\end{itemize}
\begin{center}
	\includegraphics[width=\linewidth]{images/programy_qle.png}
\end{center}

Aby uruchomić ten program, wpisz w Dashu \textcolor{ubuntu_orange}{QLE}
