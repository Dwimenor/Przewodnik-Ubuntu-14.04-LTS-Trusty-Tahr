Jednym z przydatnych bajerów jest wyświetlanie na pulpicie zawartości katalogu domowego. W~ten sposób katalogi, takie jak Pobrane, Muzyka, Dokumenty, czy Wideo, będą pod ręką na pulpicie. Aby to zrobić, otwórz edytor tekstu Gedit. Wybierz \menu{{Plik}>{Otwórz}} i wciśnij kombinację klawiszy \keys{CTRL +h}, aby wyświetlić ukryte pliki. Przejdź do katalogu .config i otwórz plik \textcolor{ubuntu_orange}{user-dirs.dirs}. Odszukaj linię:
\begin{lstlisting}
XDG_DESKTOP_DIR="$HOME/Pulpit"
\end{lstlisting}
i zmień ją na:
\begin{lstlisting}
XDG_DESKTOP_DIR="$HOME"
\end{lstlisting}

Zmiany będą widoczne po ponownym zalogowaniu do systemu.
