Program \textcolor{ubuntu_orange}{Ubuntu Tweak} jest zestawem narzędzi pozwalających usprawnić system Ubuntu. W przeciwieństwie to omawianego wcześniej Unity Tweak Tool, ten program mniej koncentruje się na wyglądzie systemu a bardziej na jego zachowaniu. Część opcji Ubuntu Tweak pokrywa się z Unity Tweak Tool (głównie tych dotyczących wyglądu środowiska), ale ma też kilka bardzo przydatnych narzedzi:

\begin{itemize}
\item \menu{{Usprawnienia}>{Logowanie}} zmiana wyglądu ekranu logowania (Greetera).
\item \menu{{Zarządzanie}>{Polecenia Aktywatorów}} edycja SzybkichList (quickList) aktywatorów na pasku Launchera.
\item \menu{{Zarządzanie}>{Menadżer typów plików}} zarządzanie powiązaniami plików z odpowiednimi programami.
\item \menu{Czyszczenie} podstawowe sprzatanie systemu i przestrzeni użytownika.
\end{itemize}

Ubuntu Tweak możesz pobrać bezpośrednio ze \href{http://ubuntu-tweak.com/}{strony autora}, lub dodać odpowiednie repozytorium i zainstalować przy pomocy menadżera pakietów:
\begin{lstlisting}[language=bash]
sudo add-apt-repository ppa:tualatrix/ppa
sudo apt-get update
sudo apt-get install ubuntu-tweak
\end{lstlisting}

Więcej o instalacji oprogramowania w Ubuntu przeczytasz w rozdziale \ref{instalacja_oprogramowania}: ,,Instalacja oprogramowania''