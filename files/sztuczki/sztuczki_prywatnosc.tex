Jednym z zastrzeżeń wobec środowiska Unity jest kwestia prywatności użytkowników. Jeśli wpisujesz cokolwiek w Dashu, to twój tekst trafia także na serwery producenta Ubuntu --- firmy Canonical. W odpowiedzi dostajesz dodatkowe wyniki wyszukiwania, np. ze sklepu Amazon. Nie wszystkim musi się podobać takie częściowe szpiegowanie poczynań użytkownika, chociaż Canonical zapewnia, iż wszystkie dane przekazywane są anonimowo.

Wyszukiwanie w źródłach sieciowych można wyłączyć. Warto to zrobić nie tylko ze względu na ochronę prywatności, ale także po to, by przyspieszyć wyszukiwanie w Dashu i zwiększyć czytelność wyników.

\noindent \menu{{Ustawienia systemu}>{Prywatność i bezpieczeństwo}>{Wyszukaj}}. Przestaw przełącznik z pozycji włączony (podświetlony), na wyłączony (szary).

Wyłączenie wyników wyszukiwania online to nie wszystko co można zrobić w kwestii prywatności użytkownika. Istnieje specjalny skrypt powłoki, który automatycznie wykona wyżej opisaną czynność a także kilka innych\footnote{Autorem skryptu jest \href{https://micahflee.com/}{Micah Lee}.}. W terminalu wykonaj następujące polecenie:
\begin{lstlisting}[language=bash]
wget -q -O - https://fixubuntu.com/fixubuntu.sh | bash
\end{lstlisting}
\begin{itemize}
\item \textcolor{ubuntu_orange}{wget} --- program do pobierania plików z internetu;
\item \textcolor{ubuntu_orange}{-q} --- opcja ,,cicha praca'' (quiet), określająca, że wget ma nie wyświetlać żadnych informacji na ekranie;
\item \textcolor{ubuntu_orange}{-O} --- opcja ,,wyjście'' (output), określająca gdzie wget ma zapisać pobrany plik;
\item \textcolor{ubuntu_orange}{-} --- myślnik po opcji ,,wyjście'' określa, że plik będzie wyświetlony na ekranie;
\item \textcolor{ubuntu_orange}{$|$} --- ,,rura'', ,,pipe''; przekierowanie wyjścia z jednego programu na wejście drugiego programu; w tym wypadku wyjściem z wgeta jest pobrany z internetu skrypt, który zostanie przekazany do programu bash;
\item \textcolor{ubuntu_orange}{bash} --- interpreter poleceń; na swoim wejściu dostał pobrany plik i teraz wykona zawarte w nim polecenia.
\end{itemize}
Pobrany i wykonany zostanie skrypt, który bardzo restrykcyjnie ustawi opcje prywatności w systemie:
\begin{itemize}
\item wyłącza wyszukiwanie w źródłach online (tak jak powyższa metoda graficzna);
\item wyłącza wtyczki wyszukiwania w Amazonie, PopularTracks, Skimlinks, eBay i sklepie z gadżetami Ubuntu;
\item blokuje połączenia z adresem productsearch.ubuntu.com we wszystkich programach zainstalowanych w systemie.
\end{itemize}