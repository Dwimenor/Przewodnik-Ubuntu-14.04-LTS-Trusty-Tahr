Wiele płyt DVD z filmami zabezpieczone jest systemem CSS (Content Scramble System), ograniczającym możliwość ich odtwarzania na nieautoryzowanych urządzeniach. W kilku krajach obchodzenie takich zabezpieczeń (nawet jeżeli legalnie zakupiłeś płytę z filmem) jest nielegalne i dlatego w~Ubuntu domyślnie nie ma oprogramowania umożliwiającego odtwarzanie takich multimediów.

Aby mieć możliwość odtwarzania zabezpieczonych płyt DVD, zainstaluj pakiet \textcolor{ubuntu_orange}{libdvdread4} (jeżeli wcześniej zainstalowałeś ubuntu-restricted-extras, to możesz pominąć ten krok):

\begin{lstlisting}[language=bash]
sudo apt-get install libdvdread4
\end{lstlisting}

\noindent Następnie w Terminalu wykonaj następujące polecenie:

\begin{lstlisting}[language=bash]
sudo /usr/share/doc/libdvdread4/install-css.sh
\end{lstlisting}

\noindent Uruchomiony w ten sposób program pobierze i zainstaluje brakujące oprogramowanie. Odtwarzanie zabezpieczonych płyt DVD powinno być teraz możliwe przy pomocy większości dostępnych odtwarzaczy (np. Totem, MPlayer, VLC).
