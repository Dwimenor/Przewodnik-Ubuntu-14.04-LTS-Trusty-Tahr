Wiele płyt DVD z filmami zabezpieczonych jest systemem CSS (Content Scramble System) ograniczającym możliwość odtwarzania takich płyt na nieautoryzowanych urządzeniach. W kilku krajach obchodzenie takich zabezpieczeń (nawet jeżeli legalnie zakupiłeś płytę z filmem) jest nielegalne i dlatego w Ubuntu nie znalazło się oprogramowanie umożliwiające odtwarzania takich multimediów.

Aby uruchomić odtwarzanie zabezpieczonych płyt DVD w Twoim komputerze zainstaluj pakiet \textcolor{ubuntu_orange}{libdvdread4}. (Jeżeli wcześniej zainstalowałeś ubuntu-restricted-extras, to nie możesz pominać ten krok)
\begin{lstlisting}[language=bash]
sudo apt-get install libdvdread4
\end{lstlisting}
\begin{itemize}
\item \textcolor{ubuntu_orange}{sudo} --- wykonuje dalsze polecenia z uprawieniami administratora systemu.
\item \textcolor{ubuntu_orange}{apt-get} --- program do zarządzania zainstalowanym oprogramowaniem.
\item \textcolor{ubuntu_orange}{install} --- informujesz apt, że chcesz zainstalować paczkę z oprogramowaniem.
\item \textcolor{ubuntu_orange}{libdvdread4} --- nazwa paczki do zainstalowania.
\end{itemize}

następnie w terminalu wykonaj następujące polecenie:
\begin{lstlisting}[language=bash]
sudo /usr/share/doc/libdvdread4/install-css.sh
\end{lstlisting}
uruchomiony w ten sposób program pobierze i zainstaluje w Twoim systemie brakujące oprogramowanie. Odtwarzanie płyt DVD powinno być teraz możliwe przy pomocy większości dostępnych odtwarzaczy (np. Totem, MPlayer, VLC).