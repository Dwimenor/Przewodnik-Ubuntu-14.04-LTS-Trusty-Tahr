Okresowe czyszczenie systemu Ubuntu nie jest potrzebne. Pewne rzeczy pozostają, ale nie mają one wpływu na szybkość działania systemu. Usówanie oprogramowania może pozostawić pewne śmieci w systemie plików ale nie wpływa to na szybkość działania. Jeżeli zachodzi potrzeba uwolnienia dodatkowego miejsca na dysku twardym, to poniższe porady mogą być pomocne.

\subsubsection{Czyszczenie katalogu domowego programem Bleachbit}
Korzystając z Centrum Oprogramowania Ubuntu znajdź i zainstaluj program \textcolor{ubuntu_orange}{Bleachbit}. Program ten wyszuka niepotrzebne i zaśmiecające katalog domowy pliki, np. miniatury plików albo pamięć podręczną przeglądarek internetowych. Nie zalecamy korzystania z Bleachbita do dokonywania zmian w systemie (opcja APT). Takie rzeczy lepiej wykonać ręcznie.

\subsubsection{Czyszczenie systemu zarzadzania pakietami}
Instalacja i deinstalacja oprogramowania może pozostawić dużo plików na komputerze. Następujące polecenia usuną je z dysku twardego.
\begin{itemize}
\item Usunięcie pobranych archiwów a pakietami
\begin{lstlisting}[language=bash]
sudo apt-get clean
\end{lstlisting}
\item Usunięcie starych archiwów a pakietami
\begin{lstlisting}[language=bash]
sudo apt-get autoclean
\end{lstlisting}
\item Usunięcie nieużywanych pakietów
\begin{lstlisting}[language=bash]
sudo apt-get autoremove
\end{lstlisting}
\end{itemize}

Dodatkowo można się posłużyć pakietem \textcolor{ubuntu_orange}{deborphan}, który pomoże usunąć nieużywane pakiety. Najpierw go zainstauj:
\begin{lstlisting}[language=bash]
sudo apt-get deborphan
\end{lstlisting}
A następnie wykorzystaj w ten sposób:
\begin{lstlisting}[language=bash]
sudo deborphan --guess-all | xargs sudo apt-get -y remove --purge
\end{lstlisting}
\begin{itemize}
\item \textcolor{ubuntu_orange}{sudo} nakazuje wykonanie polecenia jako administrator.
\item \textcolor{ubuntu_orange}{deborphan} program szukający osieroconych pakietów.
\item \textcolor{ubuntu_orange}{\-\-quess-all} flaga dla programu deborphan informująca go, że ma szukać we wszystkich pakietach.
\item \textcolor{ubuntu_orange}{$\vert$} przekazanie wyniku działania (wyjścia) programu deborphan do drugiego programu (na jego wejście).
\item \textcolor{ubuntu_orange}{sudo} nakazuje wykonanie polecenia jako administrator.
\item \textcolor{ubuntu_orange}{apt-get} program zarządzający pakietami
\item \textcolor{ubuntu_orange}{-y} flaga dla apt-get informująca, że z góry zgadzamy się na wszystkie działania
\item \textcolor{ubuntu_orange}{remove} flaga dla apt-get informująca, ze ma usunąć wskazane programy
\item \textcolor{ubuntu_orange}{\-\-purge} flaga dla apt-get informująca, ze ma także usunać pozostałości (pliki konfiguracyjne, historię, ikony itp).
\end{itemize}