Okresowe czyszczenie systemu Ubuntu nie jest potrzebne. Pewne rzeczy pozostają, ale nie mają one wpływu na szybkość działania systemu. Usuwanie oprogramowania może pozostawić trochę śmieci w systemie plików, ale nie wpływa to na szybkość działania. Jeżeli zachodzi potrzeba uwolnienia dodatkowego miejsca na dysku twardym, to poniższe porady mogą być pomocne.

\subsubsection{Czyszczenie katalogu domowego programem Bleachbit}
Korzystając z Centrum Oprogramowania Ubuntu znajdź i zainstaluj program \textcolor{ubuntu_orange}{Bleachbit}. Wyszuka on niepotrzebne i zaśmiecające katalog domowy pliki, np. miniatury albo zawartość pamięci podręcznej przeglądarek internetowych. Nie zalecamy korzystania z Bleachbita do dokonywania zmian w systemie (opcja APT). Takie rzeczy lepiej wykonać ręcznie.

\subsubsection{Czyszczenie systemu zarządzania pakietami}
Instalacja i usuwanie oprogramowania może pozostawić w komputerze dużo zbędnych plików. Poniższe polecenia usuną je z dysku twardego.
\begin{itemize}
\item Usunięcie pobranych archiwów z pakietami:
\begin{lstlisting}[language=bash]
sudo apt-get clean
\end{lstlisting}
\item Usunięcie starych archiwów a pakietami:
\begin{lstlisting}[language=bash]
sudo apt-get autoclean
\end{lstlisting}
\item Usunięcie nieużywanych pakietów:
\begin{lstlisting}[language=bash]
sudo apt-get autoremove
\end{lstlisting}
\end{itemize}

Dodatkowo można się posłużyć programem \textcolor{ubuntu_orange}{deborphan}, który pomoże usunąć nieużywane pakiety. Najpierw go zainstauj:
\begin{lstlisting}[language=bash]
sudo apt-get install deborphan
\end{lstlisting}
A następnie wykorzystaj w ten sposób:
\begin{lstlisting}[language=bash]
sudo deborphan --guess-all | xargs sudo apt-get -y remove --purge
\end{lstlisting}
\begin{itemize}
\item \textcolor{ubuntu_orange}{sudo} nakazuje wykonanie polecenia jako administrator;
\item \textcolor{ubuntu_orange}{deborphan} program szukający osieroconych pakietów;
\item \textcolor{ubuntu_orange}{\-\-quess-all} polecenie dla programu deborphan, nakazujące mu szukać we wszystkich pakietach;
\item \textcolor{ubuntu_orange}{$\vert$} przekazanie wyniku działania (wyjścia) programu deborphan do drugiego programu (na jego wejście);
\item \textcolor{ubuntu_orange}{sudo} nakazuje wykonanie polecenia jako administrator;
\item \textcolor{ubuntu_orange}{apt-get} program zarządzający pakietami;
\item \textcolor{ubuntu_orange}{-y} opcja programu apt-get, określająca, że z góry zgadzamy się na wszystkie działania;
\item \textcolor{ubuntu_orange}{remove} polecenie programu apt-get, nakazujące mu usunąć wskazane programy;
\item \textcolor{ubuntu_orange}{\-\-purge} polecenie programu apt-get, nakazujące mu usunąć również pozostałości (pliki konfiguracyjne, historię, ikony, itp).
\end{itemize}
