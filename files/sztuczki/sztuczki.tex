Jeżeli podążałeś za instrukcjami zawartymi w tym Przewodniku, to masz już działający i skonfigurowany system Ubuntu. Nie oznacza to, że nie można zrobić niczego więcej. Możliwości konfiguracji są bardzo duże i opisanie ich wszystkich daleko wykracza poza zakres tego opracowania.

W niniejszym dziale zebraliśmy kilkanaście porad dotyczących Ubuntu. Są to drobnostki, które bardzo łatwo wprowadzić do swojego systemu. Nie musisz od razu stosować się do wszystkich instrukcji. Przejrzyj listę, przeczytaj opisy i zdecyduj, co cię zaintereuje. W tym rozdziale podnieśliśmy nieco poprzeczkę i w wielu miejscach opisane zostały sposoby wykonania podanych instrukcji przy pomocy konsoli. Przypuszczalnie dla wielu z was korzystanie z wiersza poleceń ma się do wygodnej pracy z systmem tak, jak alchemia do nowoczesnej chemii. Jednak nic bardziej mylnego. Wiele czynności może być wykonanych za pomocą jednego polecenia, podczas gdy graficzna alternatywa wymaga wykonania kilkunastu kliknięć i znalezienia tej jednej z wielu pozycji w menu, o którą nam chodzi. Nie przejmuj się, wszystko zostało bardzo dokładnie wytłumaczone, zarówno w sposób graficzny, jak i tekstowy.

Aby uruchomić terminal w systemie Ubuntu, wykonaj jedną z następujących czynności:
\begin{itemize}
\item Wciśnij kombinację klawiszy \keys{CTRL+ALT+t}.
\item Wciśnij kombinację klawiszy \keys{ALT+ F2} i wpisz \textcolor{ubuntu_orange}{gnome-terminal}.
\item Otwórz Dash klawiszem \keys{Super} i wpisz \textcolor{ubuntu_orange}{terminal}.
\end{itemize}

Kiedy w trakcie pracy z terminalem zostaniesz poproszony o podanie hasła w ten sposób:
\begin{lstlisting}[language=bash]
[sudo] password for <nazwa uzytkownika>:
\end{lstlisting}
\noindent wpisz swoje hasło i potwierdź klawiszem \keys{\returnwin}. Nie przejmuj się, że hasła nie widać podczas wpisywania. To normalne i zapobiega przypadkowemu podpatrzeniu, co wpisujesz.

Polecenia możesz skopiować do terminala. Zaznacz tekst, wciśnij \keys{CTRL + C}, przejdź do terminala i wciśnij \keys{CTRL + Shift + V}.

Wskazówka: poluparne polecenia kopiuj/wklej w terminalu wywoływane są z klawiszem \keys {Shift}. Możesz to zmienić w \menu{{Edycja}>{Skróty klawiszowe}}.
