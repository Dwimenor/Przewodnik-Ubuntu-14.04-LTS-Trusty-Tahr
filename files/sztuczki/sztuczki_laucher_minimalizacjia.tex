Kliknięcie w Launcherze lewym przyciskiem myszy na ikonę otwartego programu domyślnie powoduje wyświetlenie podglądu wszystkich otwartych okien tego programu. Większość programów korzysta tylko z jednego okna i w takiej sytuacji kliknięcie na nie, nie powoduje żadnej reakcji. Można zmienić zachowanie Launchera tak, aby po kliknięciu na programy z jednym oknem minimalizował je.

Opcja ta jest nieco zaszyta w systemie. Zaczniemy od instalacji odpowiedniego narzędzia. Uruchom \textcolor{ubuntu_orange}{Centrum Oprogramowania Ubuntu} i zainstaluj program \textcolor{ubuntu_orange}{Menedżer ustawień ComizConfig}.

Alternatywnie, z wykorzystaniem konsoli:
\begin{lstlisting}[language=bash]
sudo apt-get install compizconfig-settings-manager
\end{lstlisting}
\begin{itemize}
\item \textcolor{ubuntu_orange}{sudo} --- wykonuje dalsze polecenia z uprawieniami administratora systemu.
\item \textcolor{ubuntu_orange}{apt-get} --- program do zarządzania zainstalowanym oprogramowaniem.
\item \textcolor{ubuntu_orange}{install} --- informujesz apt, że chcesz zainstalować paczkę z oprogramowaniem.
\item \textcolor{ubuntu_orange}{compizconfig-settings-manager} --- nazwa paczki do zainstalowania.
\end{itemize}
Uruchom dopiero co zainstalowany program. W Dashu znajdziesz go bardzo szybko wpisując \textcolor{ubuntu_orange}{ccsm}. Włącz opcję \menu{{Ubuntu Unity Plugin}>{Zakładka Launcher}>{Minimize Single Window Applications (Unsupported)}}.

Menedżer ustawień ComizConfig jest bardzo potężnym narzędziem służącym do konfiguracji środowiska graficznego. Potrafi zdziałać cuda, ale trzeba uważać co się robi, gdyż wiele wtyczek jest ze sobą niekompatybilnych. Warto w nim pomyszkować, ale zalecamy ostrożność i dokładne czytanie komunikatów i podpowiedzi. Większość wprowadzanych tu zmian jest widoczna natychmiast i nie wymaga restartu komputera.