Jeżeli w komputerze masz dużą ilość pamięci operacyjnej (RAM), to warto się zastanowić, czy nie lepiej ją spożytkować zgodnie z zasadą \emph{RAM niewykorzystywany to RAM zmarnowany}. Ile to jest dużo RAM-u? Minimum to 2 gigabajty, ale nie zalecamy stosowania poniższych porad, jeżeli masz mniej niż 4 gigabajty.

\subsubsection{Preload i prelink}
Te dwa pakiety to daemony działające w tle. Wystarczy je zainstalować, aby rozpoczęły swoją pracę. \textcolor{ubuntu_orange}{Preload i prelink} obserwują działania użytkownika i starają się przewidzieć jakich programów może potrzebować. W praktyce sprowadza się to do wcześniejszego załadowania do pamięci operacyjnej najczęściej używanego oprogramowania (preload) oraz potrzebnych tym programom bibliotek (prelink). Aby zainstalować te dwa programy, wykonaj w Terminalu polecenie:

\begin{lstlisting}[language=bash]
sudo apt-get install preload prelink
\end{lstlisting}

Nie martw się, że zabraknie ci RAM-u. Ładowane w ten sposób programy lądują w pamięci tymczasowej (cache) i zostaną z niej usunięte w pierwszej kolejności, jeżeli aktualnie działające oprogramowanie będzie potrzebować więcej pamięci.

\subsubsection{Wykorzystanie partycji wymiany}
Jeżeli w trakcie instalacji, zgodnie z zaleceniami tego Przewodnika, utworzyłeś partycję wymiany (swap), to teraz dowiesz się paru istotnych rzeczy. Swap jest rozszerzeniem pamięci operacyjnej, rezydującym na dysku twardym komputera. Domyślnie system będzie zapisywał w swapie dane, które muszą znajdować się \emph{pod ręką}, ale niekoniecznie w RAM-ie. Jednakże dysk twardy (nawet typu SSD) jest wielokrotnie mniej wydajny, niż nawet najwolniejsze kości pamięci RAM. Kiedy dochodzi do konieczności odczytywania danych ze swapu, widac wyraźne spowolnienie pracy komputera. Na szczęście można temu zaradzić modyfikując parametr systemowy \textcolor{ubuntu_orange}{vm.swappiness}. Określa on jak intensywnie system ma korzystać z partycji wymiany i może przyjmować wartości od 0 do 100.

\begin{itemize}
\item \textcolor{ubuntu_orange}{0} oznacza, że swap będzie wykorzystywany tylko i wyłącznie w sytuacji, kiedy zabraknie pamięci operacyjnej.
\item \textcolor{ubuntu_orange}{5} oznacza bardzo małe wykorzystanie partycji wymiany.
\item \textcolor{ubuntu_orange}{60} to wartość domyślna.
\item \textcolor{ubuntu_orange}{100} włącza bardzo agresywne wykorzystanie dysku twardego.
\end{itemize}

Aby wyświetlić aktualną wartość, wykonaj w Terminalu polecenie:

\begin{lstlisting}[language=bash]
cat /proc/sys/vm/swappiness
\end{lstlisting}

Aby zmienić tę wartość, wykonaj:

\begin{lstlisting}[language=bash]
sudo sysctl vm.swappiness=5
\end{lstlisting}

Teraz możesz przystąpić do testowania. Spróbuj mocniej obciążyć system, aby sprawdzić, jak się zachowuje. Jeżeli coś poszło nie tak i dojdzie do zablokowania maszyny, to nie przejmuj się tym. Wystarczy ją zrestartować, a vm.swappiness zostanie ponownie ustawiony na wartość domyślną 60.

\noindent Aby permamentnie wprowadzić zmiany, otwórz plik z ustawieniami systemowymi:

\begin{lstlisting}[language=bash]
sudo gedit /etc/sysctl.conf
\end{lstlisting}

\noindent i na końcu, w nowej linii, dopisz:

\begin{lstlisting}[language=bash]
vm.swappiness = 10
\end{lstlisting}

\noindent Zapisz zmiany i zrestartuj komputer, aby sprawdzić, czy się zachowały.
