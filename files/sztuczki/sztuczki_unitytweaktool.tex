W \textcolor{ubuntu_orange}{ustawieniach systemu} możesz do pewnego stopnia skonfigurować wygląd i zachowanie się Unity. Jednakże łatwo dostępne opcje są bardzo ograniczone. Większa konfigurowalność wymaga zainstalowania dodatkowych narzędzi. Wspomniany wcześniej Menadżer Ustawień CompizConfig jest bardzo potężnym narzędziem konfiguracyjnym, ale mnogość opcji może przerażać. CCMS pozwala zmodyfikować nie tylko Unity, ale całość wyglądu Twojego systemu.

Prostrzym narzedziem, dedykowanym środowisku Unity jest \textcolor{ubuntu_orange}{Unity Tweak Tool}. Program możesz zainstalować za pomocą Centrum Oprogramowania Ubuntu lub przy pomocy konsoli
\begin{lstlisting}[language=bash]
sudo apt-get install unity-tweak-tool
\end{lstlisting}
\begin{itemize}
\item \textcolor{ubuntu_orange}{sudo} - wykonuje dalsze polecenia z uprawieniami administratora systemu.
\item \textcolor{ubuntu_orange}{apt-get} - program do zarządzania zainstalowanym oprogramowaniem.
\item \textcolor{ubuntu_orange}{install} - informujesz apt, że chcesz zainstalować paczkę z oprogramowaniem.
\item \textcolor{ubuntu_orange}{unity-tweak-tool} - nazwa paczki do zainstalowania.
\end{itemize}