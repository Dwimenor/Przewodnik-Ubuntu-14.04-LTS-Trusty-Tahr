Steam jest platformą cyfrowej dystrybucji gier, zarządzaną przez firmę Valve. Ani Steam, ani gry na nim dostępne, nie należą do wolnego oprogramowania. Tytuły są dodatkowo zabezpieczone technologią DRM, tak więc kupując za pośrednictwem Steama na zawsze wiążesz swój zakup z tą platformą. Jeżeli ci to nie przeszkadza i chcesz zainstalować Steama to wykonaj następujące kroki:
\begin{enumerate}
\item Otwórz stronę \href{http://store.steampowered.com/about/}{steampowered.com}.
\item Kliknij \textcolor{ubuntu_orange}{Install Steam Now}.
\item Pobrany plik steam.deb zapisz w katalogu domowym.
\item Uruchom konsolę \keys{CTRL + ALT + t} i wykonaj polecenie:
\begin{lstlisting}[language=bash]
sudo dpkg -i steam.deb
\end{lstlisting}
\begin{itemize}
\item \textcolor{ubuntu_orange}{sudo} --- wykonuje dalsze polecenia z uprawieniami administratora systemu;
\item \textcolor{ubuntu_orange}{dpkg} --- program do bepośredniego zarządzania paczkami z oprogramowaniem;
\item \textcolor{ubuntu_orange}{-i} --- polecenie programu dpkg, nakazujące zainstalować podaną paczkę;
\item \textcolor{ubuntu_orange}{steam.deb} --- nazwa paczki do zainstalowania.
\end{itemize}
\item Uruchom program Steam, aby dokończyć instalację. Zostaniesz poproszony o zapoznanie się z warunkami licencji i zalogowanie do swojego konta Steam (lub stworzenie nowego).
\end{enumerate}
