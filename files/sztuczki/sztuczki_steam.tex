Steam jest platformą cyfrowej dystrybucji gier zarzadzaną przez firmę Valve. Zarówno Steam jak i gry na nim dostepne nie są wolnym oprogramowaniem. Tytuły są dodatkowo zabezpieczone technologią DRM, tak więc kupując za pośrednictweam Steama na zawsze wiążesz swój zakup z tą platformą. Jeżeli ci to nie przeszkadza i chcesz zainstalować Steama to:
\begin{enumerate}
\item Udaj się na \href{http://store.steampowered.com/about/}{steampowered.com}
\item Kliknij na \textcolor{ubuntu_orange}{Install Steam Now}.
\item Pobrany plik steam.deb zapisz w katalogu domowym.
\item Uruchom konsolę \keys{CTRL + ALT + t} i wykonaj:
\begin{lstlisting}[language=bash]
sudo dpkg -i steam.deb
\end{lstlisting}
\begin{itemize}
\item \textcolor{ubuntu_orange}{sudo} --- wykonuje dalsze polecenia z uprawieniami administratora systemu.
\item \textcolor{ubuntu_orange}{depg} --- program do bepośredniego zarządzania paczkami z oprogramowaniem
\item \textcolor{ubuntu_orange}{-i} --- flaga dla dpkg, że ma zainstalować podaną paczkę.
\item \textcolor{ubuntu_orange}{steam.deb} --- nazwa paczki do zainstalowania.
\end{itemize}
\item Uruchom program Steam aby dokończyć instalację. Zostaniesz poproszony o zapoznanie się z warunkami licencji i zalogowanie się do swojego konta Steam (lub stworzneie nowego).
\end{enumerate}