Instalacja oprogramowania w Ubuntu odbywa się poprzez pobranie i zainstalowanie odpowiednich paczek. Są one przechowywane na centralnych serwerach, z których co jakiś czas twój komputer pobiera informacje o najnowszych wydaniach i informuje o ewentualnych aktualizacjach. Aby przyspieszyć pobieranie danych, warto wskazać konkretny serwer, z którego system ma pobierać dane. Na świecie są setki tak zwanych ,,serwerów lustrzanych'', w tym kilka w Polsce. Aby automatycznie wybrać najlepszy serwer, kliknij: \menu{{Ustawienia systemu}>{Oprogramowanie i aktualizacje}>{Serwer pobierania}>{Inny}>{Wybierz najlepszy serwer}}.

System sprawdzi wszystkie dostępne serwery i wybierze ten, do którego uzyskał najlepsze połączenie. Operacja ta może potrwać kilkanaście sekund, w zależności od szybkości twojego łącza internetowego. Operację zmiany serwera lustrzanego musisz uwierzytelnić swoim hasłem. W chwili zamykania programu zostaniesz poproszony o zaktualizowanie listy pakietów. Wykonaj także tę czynność.
