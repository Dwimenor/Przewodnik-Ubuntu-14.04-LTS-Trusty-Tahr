Instalacja oprogramowania w Ubuntu odbywa się poprzez pobranie i zainstalowanie odpowiednich paczek. paczki te są przechowywane na scentralizowanych serwerach, z których co jakiś czas twój komputer pobiera informacje o najnowszych wydaniach i informuje cię o ewentualnych aktualizacjach. Aby przyspieszyć pobieranie danych warto wskazać konkretny serwer z którego system ma pobierać dane. Na świecie są setki tak zwanych ,,serwerów lustrzanych'', w tym kilka w Polsce. Aby automatycznie wybrać najlepszy serwer: \menu{{Ustawienia systemowe}>{Oprogramowanie i Aktualizacje}>{serwer pobierania}>{inny}>{Wybierz najlepszy serwer}}.

System sprawdzi wszystkie dostępne serwery i wybierze ten do którego uzyskał najlepsze połączenie. Operacja ta może potrwać kilkanaście sekund, w zależności od szybkości twojego łącza internetowego. Operację zmiany serwera lustrzanego musisz uwierzytelnić swoim hasłem. Przy zamykaniu programu zostaniesz poproszony o zaktualizowanie listy pakietów. Wykonaj także tą czynność.
