\subsubsection{Po wybraniu opcji ,,Wypróbuj Ubuntu bez instalacji'' lub ,,Instaluj Ubuntu'' mam tylko czarny ekran}
Po pierwsze, nie spiesz się, wczytanie wszystkich danych z nośnika instalacyjnego trwa dłuższą chwilę. Jeżeli nic się nie dzieje dłużej, niż 2 -- 3 minuty, to zrestartuj komputer i uruchom instalator Ubuntu jeszcze raz. W przypadku instalatora BIOS (fioletowe tło) po wybraniu języka wcśnij \keys{F6} (inne opcje), zaznacz opcję ,,nomodeset'' klawiszem \keys{\returnwin}, wyjdź z menu klawiszem \keys{Esc} i uruchom instalator. Opcja ,,acpi=off'' też może być pomocna.

\subsubsection{Instalacja została przerwana}
Jeżeli instalacja została przerwana w trakcie kopiowania danych na dysk, to uruchom ponownie instalator i zainstaluj system od nowa.

Jeżeli instalacja została przerwana w trakcie partycjonowania (zapisywania zmian na dysku), to prawdopodobnie doszło do nieodwracalnego uszkodzenia tablicy partycji. Uruchom nośnik instalacyjny, wybierz ,,Wypróbuj Ubuntu bez instalacji'' (\textit{Try Ubuntu without installing''} i uruchom program GParted.

\subsubsection{Po instalacji Ubuntu uruchamia się tylko Windows}
Prawdopodobnie na Windowsie masz ustawione ,,Szybkie uruchamianie'' (\textit{Fast-boot}). Wyłącz tę funkcję i ponownie zainstaluj Ubuntu. Przeczytaj rozdział \ref{sec:przygotowanie_windows}: ,,Przygotowanie do instalacji''.

\subsubsection{Po instalacji Ubuntu nie można uruchomić systemu Windows}
przyczyny mogą być dwie. Albo usunąłeś system Windows z dysku twardego, albo nie został on prawidłowo rozpoznany. W pierwszym przypadku pozostaje tylko ponownie go zainstalować. Koniecznie przeczytaj rozdział \ref{sec:przygotowanie_windows}: ,,Przygotowanie do instalacji''.

W drugim przypadku przejdź do sekcji ,,Przywracanie GRUB-a''

\subsubsection{Przywracanie GRUB-a}
\label{grub_przywracanie}
Problemy z programem rozruchowym występują najczęściej:
\begin{itemize}
\item Po instalacji systemu Windows obok Ubuntu. Windows nadpisuje MBR (\textit{Master Boot Record)} dysku twardego i uniemożliwia dostęp do systemów linuksowych.
\item Po instalacji Ubuntu obok Windowsa, jeżeli ten był zahibernowany, miał włączoną opcję ,,Fast-Boot'' lub podobne.
\item Po zmianie układu partycji.
\item Po nieumiejętnych próbach edytowania GRUB-a lub zastąpienia go innym menadżerem rozruchowym.
\end{itemize}

Najprostszym sposobem naprawienia menadżera rozruchu GRUB jest skorzystanie z Boot Repair Disk. Jest to dystrybucja Linuksa typu Live (działa tylko z zewnętrznego nośnika, nie zainstalujesz jej na dysku twardym). Otwórz stronę \href{http://sourceforge.net/projects/boot-repair-cd/files/}{pobierania}, wybierz plik z obrazem zgodny z architekturą twojego procesora i zapisz go na dysku. Proces przygotowywania nośnika z Boot Repair Disk jest analogiczny do przedstawionego w rozdziale \ref{nagrywanie_obrazu}: ,,Nagrywanie pobranego obrazu''.\\
Po uruchomieniu tak przygotowanego systemu poczekaj, aż się załaduje i wybierz opcję \textcolor{ubuntu_orange}{Recomended repair}, a nastepnie spróbuj ponownie uruchomić komputer.

Jeżeli nie możesz uruchomić żadnego systemu operacyjnego, to program Boot Repair możesz zainstalować bezpośrednio na nośniku instalacyjnym Ubuntu. Podczas uruchamiania instalatora wybierz ,,Wypróbuj Ubuntu bez instalacji'' (\textit{Try Ubuntu without installing''}, następnie otwórz terminal \keys{CTRL + Alt + t} i wydaj następujące trzy polecenia:

\begin{lstlisting}[language=bash]
sudo add-apt-repository ppa:yannubuntu/boot-repair
sudo sed "s/trusty/saucy/g\" -i /etc/apt/sources.list.d/yannubuntu-boot-repair-trusty.list
sudo apt-get update
sudo apt-get install -y boot-repair && (boot-repair &)
\end{lstlisting}

\subsubsection{Ręczne przywracanie GRUB-a}
Ta metoda jest nieprzydatna, jeżeli masz zainstalowany tylko system Windows.

Jeżeli nie masz możliwości zainstalowania paczek na nośniku instalacyjnym (np. ze względu na brak połączenia z internetem), to możesz spróbować przywrócić GRUB-a ręcznie. Uruchom instalator Ubuntu w trybie ,,Wypróbuj Ubuntu bez instalacji'' (\textit{Try Ubuntu without installing}'', a następnie otwórz terminal \keys{CTRL + Alt + t}.

Zidentyfikuj partycję, na której już jest zainstalowane Ubuntu (lub inny linux). Wykonaj:

\begin{lstlisting}[language=bash]
ls /dev |grep sd
\end{lstlisting}

Otrzymasz listę dysków twardych wraz z partycjami. \textcolor{ubuntu_orange}{sda} to najprawdopodobniej twój główny dysk twardy, sdb to pendrive. Możesz się upewnić wydając polecenie:

\begin{lstlisting}[language=bash]
sudo fdisk -l
\end{lstlisting}

Wyszukaj oznaczenie głównej partycji zainstalowanego linuksa. W wielu przypadkach będzie to /dev/sda1, ale jeżeli masz zainstalowany obok inny system (np. Windows), to może być inaczej. W dalszej częsci tego tekstu zakładamy, że jest to /dev/sda1.

Zamontuj partycję zainstalowanego systemu oraz inne niezbędne rzeczy:
\begin{lstlisting}[language=bash]
sudo mount /dev/sda1 /mnt
sudo mount --bind /dev /mnt/dev
sudo mount --bind /dev/pts /mnt/dev/pts
sudo mount --bind /proc /mnt/proc
sudo mount --bind /sys /mnt/sys
\end{lstlisting}

Przełącz się na nowy system plików:

\begin{lstlisting}[language=bash]
sudo chroot /mnt
\end{lstlisting}

Zainstaluj GRUB-a. Zwróć uwagę, że instalujemy go na dysku (sda), a nie partycji (sda1).

\begin{lstlisting}[language=bash]
grub-install /dev/sda
update-grub
\end{lstlisting}

Wyjdź i posprzątaj po sobie:

\begin{lstlisting}[language=bash]
exit
sudo umount /mnt/sys
sudo umount /mnt/proc
sudo umount /mnt/dev/pts
sudo umount /mnt/dev
sudo umount /mnt
\end{lstlisting}

Teraz ponownie uruchom komputer.
\clearpage
