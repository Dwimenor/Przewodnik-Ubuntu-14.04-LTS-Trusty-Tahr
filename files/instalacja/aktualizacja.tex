Jeżeli masz zainstalowaną wcześniejszą wersję Ubuntu to nie musisz na nowo instalować całego systemu. W Ubuntu został wbudowany mechanizm aktualizacji dystrybucji, jednak ma on pewne ograniczenia. Zaktualizować można tylko bezpośrednio z poprzedniego wydania (13.10) lub poprzedniego LTS-a (12.04). Zwróć uwagę, że jeżeli instalowałeś dodatkowe repozytoria PPA to mogły one jeszcze nie zostać zaktualizowane. Na czas aktualizacji wydania najlepiej jest wyłączyć te PPA.

Przed przystąpieniem do aktualizacji wydania zainstaluj wszystkie zwyczajne aktualizacje i ponownie uruchom komputer. Zrób też kopię zapasową ważnych danych.

Uruchom Menadżer Aktualizacji w trybie aktualizacji wydania. Wciśnij \keys{Alt + F2} i wpisz \textcolor{ubuntu_orange}{update-manager -d}. Uruchom znaleziony program i poczekaj aż przetworzy wszystkie repozytoria. Jeżeli znajdzie możliwość aktualizacji do nowszego wydania Ubuntu to zaproponuje instalację. Jeżeli nie ma takiej możliwości, kliknij na \textcolor{ubuntu_orange}{Ustawienia\ldots}, przejdź na zakładkę \textcolor{ubuntu_orange}{Aktualizacje} i zaznacz \textcolor{ubuntu_orange}{Wstępne wydania aktualizacji (proposed)}. Zapisz zmiany i spróbuj ponownie.

Aktualizacja wydania jest bardzo czasochłonna. Konieczne jest pobranie wszystkich nowych paczek (około 900 megabajtów)i ich zainstalowanie. Na starszym sprzęcie może to trwać od pół do półtorej godziny. Czysta instalacja jest znacznie szybsza.