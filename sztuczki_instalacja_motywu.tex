Motyw graficzny systemu można modyfikować, lub skorzystać z gotowców. Domyślnie w Ubuntu zainstalowane są trzy motowy: Ambiance (ciemny), Radiance (jasny) uraz High Contrast (wysoki kontrast). Zmiana motywu graficznego odbywa się w \menu{{Ustawienia systemu}>{Wygląd}>{Motyw}}.

Aby pobrać i zainstalować nowy motyw udaj się na stronę \href{http://gnome-look.org/}{Gnome-Look.org} i z panelu po lewej stronie wybierz opcję GTK 3.x. Wybierz interesujący cię motyw i kliknij na przycisk \textcolor{ubuntu_orange}{download}. Otwórz pobrany plik w Menedżerze Achiwów File-Roller. Zobaczysz ekran podobny do tego:
\begin{center}
\includegraphics[width=\linewidth]{images/programy_fileRoller.png}
\end{center}
Większość motywów pakowana jest w paczki po kilka sztuk różniących się kolorystyką. Wybierz jeden (lub wszystkie) i kliknij na przycisk \textcolor{ubuntu_orange}{Rozpakuj}. W otwartym oknie musisz wskazać gdzie mają zostać rozpakowane pliki. Motywy graficzne przechowywane są w katalogu domowym, w podkatalogu .themes. Nie masz jeszcze tego katalogu, więc upewnij się iż ekran został ustawiony na katalogu domowym i kliknij na przycisk \textcolor{ubuntu_orange}{Utwórz katalog} znajdujący się w prawym górnym rogu okna. Jako nazwę katalogu podaj .themes (z kropką na początku). Utworzy to ukryty katalog przeznaczony na motywy graficzne. Wypakuj zawartość pobranego z internetu pliku do nowo utworzonego katalogu.

Teraz należy poinformować system, że ma używać nowego motywu graficznego. Uruchom \textcolor{ubuntu_orange}{Unity Tweak Tool} a następnie z wiersza \textcolor{ubuntu_orange}{Appearance} wybierz \textcolor{ubuntu_orange}{Theme}. Na wyświetlonej liście wymienione są wszystkie motywy zainstalowane w systemie. Kliknięcie na nazwę motywu automatycznie wprowadza zmiany.