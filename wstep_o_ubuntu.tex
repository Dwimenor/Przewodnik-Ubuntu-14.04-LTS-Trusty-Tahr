Ubuntu jest kompletnym systemem operacyjnym utrzymywanym i rozwijanym przez firmę Canonical. Pierwsza werja ukazała się w 2004 roku i przez 14 lat system ten zdobył rzesze fanów. Ubuntu, wraz ze swoimi odmianami jest najpopularniejszą na świecie dystrybucją Linuksa. Samo słowo \emph{Ubuntu} w języku afrykańskiego plemienia Zulusów oznacza "człowieczeństwo wobec innych", potocznie tłumaczone na "Linux dla ludzi".

Ideą systemu Ubuntu jest dostarczenie użytkownikowi kompletnego systemu operacyjnego, zawierającego wszystko co mu niezbędne do pracy a jednocześnie umożliwiające mu swobone korzystanie i modyfikowanie systemu. Wybierając Ubuntu nie musisz się martwić tym, czy twój procesr nie ma przypadkiem za wiele rdzeni co wymagałoby zakup innej licencji na system komercyjny. Nie musisz się przejmować tym, że w firmie masz dziesięc komputerów a twoja licencja na pakiet biurowy pozwala na instalację jedynie na szejściu. To jak i do czego wykorzystasz system i oprogramowanie zależy wyłącznie od ciebie.

Ubuntu pozwala także na daleko idące modyfikacje systemu. Kod źródłowy jest otwarty, co pozwala każdemu na głęboką ingerencję w system. Jednak Ubuntu nie jest przewidziane tylko dla komputerowych czarodziejów. Każdy może dowolnie dostosować swój system do własnych potrzeb i upodobań, czy to metodą \emph{zrób to sam} czy poprzez wykorzystanie głębokich zasobów oferowanych przez społeczność.

Największą siłą napędową Ubuntu jest społeczność skupiona wokół tego systemu. Dodatki zmieniające wygląd systemu, nowe ikony i grafiki, dźwięki systemowe, tłumaczenia, całe zestawy oprogramowania - to wszystko i wiele innych czeka na ciebie.
