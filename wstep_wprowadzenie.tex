\noindent Witaj w \emph{Przewodniku po Ubuntu Linux 14.04 Trusty Thar!}

Niniejszy dokument pomoże ci zainstalować oraz skonfigurować system operacyjny Ubuntu. Przewodnik obejmuje każdy etap procesu zmiany systemu, od przygotowania twoich plików i ustawień po instalowanie oraz używanie twojej świeżo zainstalowanej kopii Ubuntu.

Przewodnik ten został napisany z myślą o osobach nieposiadających wiedzy technicznej, a większość terminów technicznych opatrzono stosownymi objaśnieniami. Przewodnik zadaje także kłam mitowi, że użytkowanie Linuksa wiąże się z koniecznością wpisywania niezrozumiałych komend w konsoli. Cały tekst został przygotowany z myślą o wykorzystaniu graficznych narzędzi dostarczanych wraz z systemem.

Mamy nadzieję, że czytając ten przewodnik bezproblemowo zainstalujesz Ubuntu na swoim komputerze i będziesz zadowolony mogąc korzystać z darmowego oraz otwartego systemu operacyjnego.

Wersja Ubuntu, która została opisana w tym poradniku, nosi nazwę Ubuntu GNU/Linux 14.04 LTS Trusty Thar, co oznacza:
\begin{description}
\item[Ubuntu] -  nazwa całej serii systemów operacyjnych wydawanych przez firmę Canonical.
\item[GNU/Linux] - system bazuje na jądrze Linuksa i wykorzystuje oprogramowanie GNU.
\item[14.04] - jest to wersja z kwietnia (04) 2014 roku.
\item[LTS] - jest to wersja o przedłużonym wsparciu technicznym, a poprawki będą wydawane do 2019 roku).
\item[Trusty Thar] - nazwa kodowa tego wydania.
\end{description}
\clearpage
