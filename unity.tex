Przy pierwszym uruchomieniu systemu, zaraz po instalacji systemu, można zauważyć, że wygląd Ubuntu wcale aż tak bardzo nieróżni się od wyglądu systemów Microsoft Windows lub OS X. Wszystkie systemy łączy ta sama idea graficznego interfejsu użytkownika (\textit{GUI: Graphical User Interface}), gdzie użytkownik systemu za pomocą myszy kontroluje większość zadań począwszy od zwykłego poruszania się po systemie, otwierania dostępnych aplikacji, a także tworzenia, usuwania i przenoszenia plików i katalogów.

W Ubuntu, domyślnym środowiskiem graficznym jest Unity.