\begin{wrapfigure}{R}{0.5\textwidth}
		\includegraphics[width=\linewidth]{images/pierwsze_uruchomienie_grub.png}
\end{wrapfigure}

Po zainstalowaniu systemu Ubuntu twój komputer został zresetowany, zostałeś też poproszony o usunięcie nośnika instalacyjnego (pendrive, płyta DVD) z napędu. Jeżeli to wykonałeś to przy ponownym uruchomieniu komputera powinieneś zobaczyć ekran bardzo podobny do tego. Jest to GRUB (Grand Unified Bootloader), program rozruchowy zajmujący uruchomieniem systemu operacyjnego. Korzystając z GRUBA możesz wybrać, który system operacyjny ma zostać uruchomiony. Korzystając z klawiszy kursora na klawiaturze podświetl odpowiednią opcję i wciśnij Enter.

Jeżeli Ubuntu to jedyny system operacyjny zainstalowany na twoim komputerze to menu GRUBa nie wyświetli się. Zamiast niego Przez około sekundę widoczny będzie ciemnofioletowy ekran a następnie zostanie uruchomiony system Ubuntu. Aby w takiej sytuacji wejść do menu GRUBa wciśnij klawisz shift kiedy fioletowy ekran jest widoczny.

\textbf{Opcje zaawansowane dla systemu Ubuntu} to zestaw dodatkowych programów naprawczych i diagnostycznych dla systemu Ubuntu. Zostały one szerze opisane w rozdzialne \ref{Rozwiązywanie problemów}

\textbf{Memorytest} to program służący do testowania pamięci operacyjnej komputera (RAM).

Opcja \textbf{Windows 7 Loader} uruchomi system operacyjny Windows 7.

\begin{flushright}
Wybierz \textbf{Ubuntu}, wciśnij klawisz Enter aby uruchomić Ubuntu.
\end{flushright}
\clearpage