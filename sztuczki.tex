Jeżeli podążałeś za instrukcjami zawartymi w tym Przewodniku to masz już działający i skonfigurowany system Ubuntu. Nie oznacza to, że to wszystko co można zrobić w systemie. Mozliwości konfiguracji są bardzo duże i opisanie ich wszystkich daleko wykracza poza zakres tego przewodnika.

W niniejszym rozdziale zebraliśmy kilkanaście porad dotyczących Ubuntu. Są to drobnostki, które bardzo łatwo wprowadzić do swojego systemu. Nie musisz od razu instalować wszystkich. Przejżyj listę, przeczytaj opisy i zdecyduj co cię zaintereuje. W tym rozdziale podnieśliśmy nieco poprzeczkę i w wielu miejscach podane zostały sposoby wykonania zawartych instrukcji przy pomocy konsoli. Dla wielu z was korzystanie z wiersza poleceń ma się do wygodnej pracy z systmem jak alchemia do nowoczesnej chemii. Jednak nic bardziej mylnego. Wiele czynności może być wykonanych za pomocą jednego polecenia, podczas gdy graficzna alternatywa wymaga wykonania kilkunastu kliknięć i znalezienia tej jednej z wielu pozycji w menu o którą nam chodzi. Nie przejmuj się, wszystko zostało bardzo dokładnie wytłumaczone, zarówno w sposób graficzny jak i tekstowy,

Aby uruchomić terminal w systemie Ubuntu wykonaj jedną z następujących czynności:
\begin{itemize}
\item Wciśnij kombinacje klawiszy \keys{CTRL+ALT+t}.
\item Wciśnij kombinacje klawiszy \keys{ALT+ F2} i wpisz \textcolor{ubuntu_orange}{gnome-terminal}.
\item Otwórz Dasha klawiszem \keys{Super} i wpisz \textcolor{ubuntu_orange}{terminal}.
\end{itemize}
\clearpage
Kiedy w trakcie pracy z terminalem zostaniesz poproszony o podanie hasła w ten sposób:
\begin{lstlisting}[language=bash]
[sudo] password for <nazwa uzytkownika>:
\end{lstlisting}
\noindent Wpisz swoje hasło o potwierdź klawiszem \keys{\returnwin}. Nie przejmuj się, ze nie widać hasła podczas wpisywania. To normalne i zapobiega przypadkowemu podpatrzeniu przez kogoś co wpisujesz.

Polecenia możesz skopiować do terminala. Zaznacz tekst, wciśnij \keys{CTRL + C}, przejdź do terminala i wciśnij \keys{CTRL + Shift + V}

Wskazówka: Poluparne polecenia kopiuj/wklej w terminalu wywoływane są z klawiszem \keys {Shift}. Możesz to zmienić w \menu{{Edycja}>{Skróty klawiszowe}}