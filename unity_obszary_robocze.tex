Obszary robocze, powszechnie zwane wirtualnymi pulpitami, pozwalają w łatwy sposób zarządzać i grupować programy dzięki czemu, przy dużej liczbie otwartych okien w łatwy sposób możemy zredukować zatłoczenie i usprawnić nawigację po naszym pulpicie. Dla przykładu, na jednym obszarze roboczym możemy przeglądać zasoby Internetu, a na drugim lub trzecim obszarze roboczym możemy mieć uruchomiony program do przeglądania multimediów oraz tworzyć dokument w programie biurowym.

Domyślnie Ubuntu daje dostęp do czterech obszarów roboczych, ale zaraz po instalacji ta opcja nie jest włączona. Aby móc korzystać z dodatkowych obszarów roboczych kliknij na ikonę Systemową \includegraphics[scale=1]{images/ikony_zasilanie.png} na panelu menu (w prawym górnym rogu) i wybierz "Ustawienia systemu". Z wiersza "Osobiste" wybierz "Wygląd". Przejdź na zakładkę "Zachowanie" i zaznacz opcję "Uaktywnij obszary robocze".

\begin{wrapfigure}[4]{l}{0.05\textwidth}
                \includegraphics[width=0.075\textwidth]{images/ikony_obszary_robocze.png}
\end{wrapfigure}

Po zaznaczeniu dodatkowych obszarów roboczych, na pasku Launchera pojawi się dodatkowa ikona, która to służy do przełączania się pomiędzy obszarami roboczymi: \textbf{Przełącznik obszarów roboczych}. Domyślnie w Ubuntu masz do dyspozycji cztery wirtualne pulpity, ułożone w siatkę 2x2. Po naciśnięciu na ikonce obszarów roboczych na naszym ekranie pojawia się siatka czterech pulpitów.\\
Pomiędzy obszarami roboczymi możesz się też przemieszczać za pomocą następujących skrótów klawiszowych:
\begin{itemize}
\item CTRL + ALT + $\rightarrow$ przemieszcza pulpit na obszar roboczy po prawej.
\item CTRL + ALT + $\leftarrow$ przemieszcza pulpit na obszar roboczy po lewej.
\item CTRL + ALT + $\uparrow$ przemieszcza pulpit na obszar roboczy na górze.
\item CTRL + ALT + $\downarrow$ przemieszcza pulpit na obszar roboczy po na dole.
\end{itemize}
\clearpage