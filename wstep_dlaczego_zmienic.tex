\subsubsection{Jest stabilny}
Ubuntu bazuje na słynącym ze stabilności systemie Debian GNU/Linux. Zapomnij o błędach krytycznych i zawieszaniu się komputera, przyzwyczaj się natomiast do niezawodnego systemu, który po prostu działa. Standardy jakości Debiana są bardzo wysokie i do ostatecznej wersji tego systemu nie trafi nic, co mogłoby nagle się popsuć. Jeśli instalujesz pod Ubuntu jakąś aplikację, możesz mieć pewność, że została już ona przetestowana przez tysiące ludzi rozsianych po całym świecie.
\begin{itemize}
\item Debian GNU/Linux jest tak stabilny, że pod jego kontrolą pracują najważniejsze systemy komputerowe świata, wliczając w to superkomputery oraz serwery wielkich portali internetowych.
\item Poprawki eliminujące znalezione błędy trafiają do systemu na bieżąco i nie trzeba na nie czekać miesiącami.
\item Każdy może zgłaszać znalezione błędy i śledzić proces ich naprawiania.
\end{itemize}
\subsubsection{Jest bezpieczny}
Ubuntu prezentuje zupełnie inne podejście do zagadnienia bezpieczeństwa niż inne systemy operacyjne. Tutaj bezpieczeństwo wynika z samej konstrukcji systemu,nie jest natomiast rezultatem nakładania na niego kolejnych łatek i dodatkowych warstw ochronnych. System jest bezpieczny, ponieważ likwiduje się przyczynę ewentualnych problemów, a nie leczy objawy. Co więcej, błędy, które mogłyby mieć negatywny wpływ na bezpieczeństwo użytkownika, naprawiane są niemal natychmiast. Nierzadko zdarza się, że od momentu wykrycia luki do chwili instalacji stosownej poprawki na milionach komputerów mija mniej niż doba.
\begin{itemize}
\item Ponieważ różne dystrybucje Linuksa cechują się wysokim poziomem bezpieczeństwa, to właśnie te systemy operacyjne można znaleźć na bardzo wielu serwerach sieciowych.
\item Wprowadzanie poważniejszych zmian w systemie pociąga za sobą konieczność podania hasła administratora. Ubuntu jest dzięki temu zabezpieczone zarówno przed potencjalnymi intruzami, jak i przed przypadkowym naruszeniem zasad bezpieczeństwa przez samego użytkownika.
\end{itemize}
\subsubsection{Jest łatwy w użyciu}
Słowo Ubuntu tłumaczy się jako ”\textit{człowieczeństwo wobec innych}”,  także “Linux dla ludzi”. Użytkowane przez ciebie programy zostały zaprojektowane w taki sposób, by nie były bardziej skomplikowane, niż to konieczne. To wcale nie znaczy, że Ubuntu ma ograniczone możliwości czy brakuje mocy - wręcz przeciwnie, pulpit Ubuntu jest pełen innowacyjnych funkcji.
\begin{itemize}
\item Komunikaty są sformułowane w jednoznaczny sposób, więc będziesz potrzebował przeczytać je tylko raz.
\item Aplikacje są ułożone tak, aby było je łatwo znaleźć.
\item Programy mają schludny i nowoczesny interfejs, dzięki któremu łatwiej będzie ci skupić się na czekających cię zadaniach.
\end{itemize}
\subsubsection{Jest międzynarodowy}
Nieważne gdzie mieszkasz i w jakim języku mówisz - możesz być pewien, że Ubuntu będzie się komunikowało z każdym użytkownikiem w najbardziej zrozumiały dla niego sposób. Dostęp do różnych wersji językowych jest bardzo prosty, a zmiana języka systemu ogranicza się do kilku kliknięć.
Oprócz  tłumaczeń obejmujących między innymi komunikaty systemowe, interfejs użytkownika czy menu poszczególnych aplikacji, Ubuntu oferuje również pełny wybór zestawów znaków i metod wprowadzania tekstu, możesz więc porozumiewać się ze swoim komputerem w dowolnie wybranym języku.
\begin{itemize}
\item Tłumaczenia są tworzone przez ochotników z całego świata.
\item Możesz samemu zaangażować się w tłumaczenia, korzystając z internetowej usługi Launchpad.
\item Aplikacja “Języki” pozwala szybko i wygodnie instalować nowe paczki językowe .
\end{itemize}
\subsubsection{Jest dostępny}
Świeżo zainstalowane Ubuntu wyposażone jest w szereg narzędzi poprawiających łatwość dostępu - lupę, program czytający informacje pojawiające się na ekranie oraz klawiaturę ekranową. Projekt Ubuntu posiada Zespół Dostępności, który zajmuje się wyłącznie tym, aby Ubuntu stawało się coraz bardziej dostępne dla każdego.
\begin{itemize}
\item Użytkownik może korzystać z ułatwień dostępu przez cały czas - od procesu instalacji począwszy, na codziennym użytkowaniu skończywszy.
\end{itemize}
\subsubsection{Jest wolny}
Ubuntu jest wolne i otwarte. Za instalację i użytkowanie tego systemu operacyjnego nigdy nie będziesz musiał zapłacić ani grosza. Nikt nie zabroni Ci również  modyfikowania, używania i rozprowadzania aplikacji wchodzących w skład Ubuntu. Nie musisz się zastanawiać nad tym, czy możesz wykorzystywać dany program, czy też jego licencja pozwala na przykład jedynie na ściśle określone zastosowania. W przypadku korzystania z Ubuntu nie ma takich ograniczeń - dysponujesz całkowitą wolnością w kwestii wykorzystywania i modyfikowania systemu oraz zawartego w nim oprogramowania.
Co więcej, zachęcamy cię do takiego postępowania! To oznacza, że zaoszczędzisz na oprogramowaniu, ale to nie wszystko - pamiętaj także, że jest ono całkowicie transparentne i otwarte na analizę. To pozwala szybciej wykrywać problemy związane z bezpieczeństwem, uniemożliwia ukrywanie przed niczego nieświadomym użytkownikiem przykrych niespodzianek, a na dodatek masz możliwość samodzielnego  dokonywania zmian w Ubuntu.
\begin{itemize}
\item Jeśli tylko posiadasz odpowiednią wiedzę techniczną, możesz samemu modyfikować swoje ulubione aplikacje.
\item Ubuntu może używać absolutnie każdy.
\end{itemize}
\subsubsection{Jest społecznościowy}
Społeczność to opoka, na której opiera się Ubuntu. Bez owej społeczności Ubuntu nie byłoby światowej klasy systemem operacyjnym, jakim jest w 2014 roku. Społeczność jest nierozłącznie związana z sukcesem Ubuntu i to właśnie ona zajmuje się wieloma rzeczami, od dostarczania tłumaczeń, testowania nowych wydań i zapewniani wsparcia, aż po pisanie nowego oprogramowania i rozwiązywanie problemów,  Każdy może pomóc w takim zakresie, w jakim potrafi i ma ochotę. Również i ty możesz pomóc kształtować kierunek rozwoju Ubuntu i ulepszać oprogramowanie dla ludzi z całego świata.
\begin{itemize}
\item Każdy może wnieść swój wkład w rozwój Ubuntu.
\item Ubuntu skupia ludzi posiadających bardzo różne zainteresowania. Programiści nie są jedynymi wybrańcami, którzy mają szansę zobaczyć efekty swojej pracy na milionach komputerów. Takie same możliwości mają graficy tworzacy tapety, muzycy komponujący dźwięki systemowe, designerzy projektujący ikony i zajmujący się wyglądem aplikacji, tłumacze dbający o to, by Ubuntu było dostępne w tylu językach, a także wiele, wiele innych osób.
\item Kodeks Postępowania Ubuntu i Rada Społeczności pomaga przewodzić społeczności i zapewnia każdemu możliwość przedstawienia swoich racji.
\end{itemize}
\clearpage
